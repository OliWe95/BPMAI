\section{Stanford Parser}
- Geschichte der Natural language process group\par
- Java Architektur \par 
- auch andere Sprachen \par

Im folgenden soll der Stanford Parser eingeführt werden.Dabei handelt es sich um ein quelloffenes Tool zur Analyse von Natürlicher Sprache, was auch als \ac{NLP} bezeichnet wird. Das grundlegende Prinzip des Stanford Parsers (eigentlich Stanford CoreNLP) besteht darin, den zu analysierenden Text in einzelne Elemente, also etwa Worte, zu zerlegen und diese mit Meta-Annotationen zu versehen. Zu diesem Zweck steht eine individuell erweiterbare Menge an Funktionen zur Auswahl. Diese Funktionen basieren auf unterschiedlichen Herangehensweisen, weswegen manche bereits eine lexikalische Analyse beinhalten, andere jedoch nicht.\par 
Die Im Standard enthaltenen Funktionen werden weiterhin aufgeführt.

\subsection{Part-of-Speech}
Als \ac{POS} wird die Art eines Wortes bezeichnet. Ein Wort kann beispielsweise ein Nomen, Verb oder Adjektiv sein. Der Stanford Parser verfügt über eine Funktionalität, die \ac{POS} Annotationen zu den einzelnen Worten des analysierten Texts hinzufügt.\par
Zu diesem Zweck werden vorhergehende und folgende Wörter im jeweiligen Satz betrachtet. Realisiert wird diese Funktionalität über ein dependency Network. (\cite[vgl.][1]{POSTAGGER})

\subsection{Named-Entity-Recognition}
Ein Bestandteil natürlicher Sprache ist die Erwähnung von Namen etwa bestimmter Orte, Personen oder auch Unternehmen. Daher verfügt das CoreNLP Kit über die \ac{NER}. Mit Hilfe dieser Funktionalität können derartige Wörter erkannt und in Bezug auf die bezeichnete Entität klassifiziert werden. Beispielsweise könnte ein Wort mit der Annotation "'Organization"' versehen werden, was darauf hinweist, dass es sich dabei um ein Unternehmen handelt.

\subsection{Sentiment Analysis}
Neben reinem Inhalt verfügt ein Satz in natürlicher Sprache auch über eine Stimmung. Diese äußert sich vorrangig anhand der Auswahl der Worte und deren Konnotation. Das CoreNLP Kit bietet daher die Sentiment Analysis. Hiermit können Sätze bezüglich ihrer Stimmung als "'negativ"', "'neutral"' oder "'positiv"' annotiert werden.\par
Umgesetzt wird diese Funktionalität unter Anwendung eines rekursiven Neuronalen Netzes, also eines Deep Learning Algorithmus. Zum Training des Netzes wurde die "'Stanford Sentiment Treebank"' verwendet,eine Sammlung von Begriffen versehen mit der jeweiligen Konnotation (\cite[vgl.][1]{SOCHERSENTIMENT}).

\subsection{Coreference Resolution}
Ein Text in natürlicher Sprache bezieht sich auf Entitäten nicht nur anhand deren eigentlicher Bezeichnung. In der Regel werden diese Bezeichnungen in derartigen Bezügen auch durch Artikel oder Pronomina ersetzt. Dies kann insbesondere auch satzübergreifend geschehen. Das CoreNLP Kit ermöglicht über die Coreference Resolution eine Auflösung der Beziehungen zwischen den bezeichneten Entitäten und ihren jeweiligen substitutiven Elementen. 
