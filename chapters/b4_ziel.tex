\section{Ziel}

Ziel der Arbeit ist es, eine Analyse der bestehenden \ac{NLP} Tools und des State-Of-The-Art nach Friedrich durchzuführen (\cite[vgl.][]{FRIEDRICH1}). Diese dient als Vorbereitung für die Weiterentwicklung des bestehenden Ansatzes und schafft Basiswissen, dass zur eigenen Implementierung eines \textit{\ac{T2P}} Tools, das einen Text in natürlicher Sprache in eineen Graphen überführt, notwendig ist.

Zunächst werden in Kapitel 2 die grundlegenden Begriffe \textit{Business Process Modelling}, \textit{Natural Language Processing} und \textit{Text2Process} geklärt. In Kapitel 3 folgt eine ausführliche Analyse der \ac{NLP} Tools \textit{StanfordCoreNLP}, \textit{WordNet} und \textit{FrameNet}, auf die auch Friedrich zurückgreift. Kapitel 4 befasst sich mit der Analyse des State-Of-The-Art. Der Ansatz wird schrittweise betrachtet und gegebenenfalls Schwachstellen identifiziert. Das abschließende Kapitel 5 fasst die Ergebnisse zusammen und gibt einen Ausblick auf die Zukunft des \ac{T2P} unter Berücksichtigung der Forschungsfortschritte in der \ac{AI}-Teildisziplin Machine Learning.

Die Arbeit beschränkt sich auf das Teilproblem \textit{Text-2-WorldModel} und \ac{NLP} in englischer Sprache. Die Übertragbarkeit der Vorgehensweise in andere Sprachen ist nicht Teil der Arbeit. Das zweite Teilproblem, \textit{WorldModel-To-BPM} wird in der Seminararbeit nicht diskutiert.
