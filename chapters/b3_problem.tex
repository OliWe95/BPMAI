\section{Problemstellung}

Die automatische Überführung eines Textes in natürlicher Sprache in einen Graphen ist kein triviales Problem. Unter den bisher entwickelten Ansätzen gilt die Methodik nach Fabian Friedrich (\cite[vgl.][]{FRIEDRICH1}) als State-Of-The-Art (\cite[vgl.][]{RIEFER}). Im ersten Schritt der Überführung wird der Text mittels \textit{\ac{NLP}-Techniken} in eine geeignete Meta-Datenstruktur, ein \textit{World Model}, überführt. Der Inhalt dieser Seminararbeit widmet sich diesem ersten Teilproblem, der Textverarbeitung von natürlicher Sprache.
\par
Zur Vorbereitung auf eine eigene, verbesserte Implementierung der Verarbeitungsschritte nach Friedrich ist ein umfassendes Verständnis der gängigen \ac{NLP} Tools und eine Analyse des bestehenden State-Of-The-Art notwendig.



 