\section{Zusammenfassung und Fazit}
Diese Arbeit befasste sich ausführlich mit der Darstellung und den An\-wen\-dungs\-mög\-lic\-hei\-ten von \ac{AI} Tools auf die Prozessmodell Generierung aus Beschreibungen in natürlicher, englischer Sprache.\par Zunächst wurden die fachlichen Grundlagen zu \ac{BPM}, \ac{NLP} und \ac{T2P} erläutert. Aufbauend darauf wurden anschließend die \ac{NLP} Tools eingeführt. Das Stanford CoreNLP wurde als Vereinigung vieler \ac{NLP} Funktionen beschrieben, die eine einheitliche Java Schnittstelle bietet. Weiterhin wurden die beiden lexikalischen Ressourcen WordNet und FrameNet erläutert die mit Synonymen bzw. Frames auf Basis unterschiedlicher theoretischer Ansätze organisiert sind.
Die Betrachtung des State-Of-The-Art Ansatzes zeigte letztlich, dass die \ac{NLP} Tools eine entscheidende Rolle spielen. Daher ergeben sich mit der Weiterentwicklung der betrachteten \ac{NLP} Tools an manchen Stellen des Verfahrens neue Möglichkeiten.\par
Folglich wurden die Grundlagen und Tools zum Thema Text-2-WorldModel erläutert sowie der State-Of-The-Art Ansatz analysiert. Das in Kapitel \ref{SUBSEC:ZIEL} formulierte Ziel wurde somit erreicht.

\section{Ausblick}
Letztlich soll auf die Frage eingegangen werden, wie sich der State-Of-The-Art Ansatz noch weiter verbessern lässt. Als Benchmark hierfür könnte die in Abschnitt \ref{SUBSEC:PROBLEMFELDER} dargelegte Korrektheit des Ansatzes bei Anwendung auf bestimmte Testdaten von durchschnittlich 76\% dienen. Eine naheliegende Vorgehensweise zur Verbesserung wäre zum einen die Optimierung der Auswahl der \ac{NLP} Tools, wie ebenfalls in \ref{SUBSEC:PROBLEMFELDER} dargelegt. Zum anderen könnten auch die internen Algorithmen des Ansatzes, wie etwa die Extraktion der Actor, Ressourcen oder Actions mit komplexerer Logik verbessert werden. Eine hierzu Alternative Vorgehensweise soll weiterhin kurz motiviert und erläutert werden.\par  
Bei Friedrichs State-Of-The-Art Ansatz handelt es sich um einen regelbasierten Ansatz, der sich auch der Anwendung von Heuristiken bedient. Ein Beispiel hierfür stellen die in Abschnitt \ref{subsec:TextLevel} erläuterten Marker dar, deren Erkennung an statische Listen gekoppelt ist. Natürliche Sprache kann jedoch nicht vollständig an Regeln festgemacht werden, denn es bleibt immer ein Interpretationsspielraum. Folglich Bedarf es auch für die \ac{T2P} an menschlicher Interpretation. \par
Die Machine Learning Methode \textit{Supervised Learning} wird bei fachlich ähnlichen Problemen bereits angewendet und verspricht effektive Resultate (\cite[vgl.][2]{BPMML}). \textit{Supervised Learning} bezeichnet das Herleiten eines Zusammenhangs durchLlernen aus Trainingsdaten in Form von Input-Output Paaren(\cite[vgl.][695]{AIMODERN}). So könnte Beispielsweise die natürliche Prozessbeschreibung oder die bereits durch \ac{NLP} Tools getaggte Beschreibung als Input und das Worldmodel als Output gewählt werden. Es wären folglich manuell erstellte Datenpaare dieser Art notwendig, um einen \ac{T2P} Algorithmus auf Basis von Supervised Learning herleiten zu können.\par
Ein entscheidender Nachteil von Supervised Learning ist jedoch in diesem Kontext, dass sehr große Datenmengen dieser speziellen Trainingsdaten benötigt werden, die für das betrachtete Problem nicht existieren. Als Referenz dient ein vergleichbares Projekt, die Penn Treebank, die bereits in Kapitel \label{subsec:pos} erwähnt wurde. Diese wurde über 8 Jahre hinweg erstellt (\cite[vgl.][1]{PENNTREEBANK}). Der Aufbau einer solchen Datenbank für Prozessbeschreibungen und Prozessmodelle wäre folglich auch sehr aufwendig, könnte jedoch dazu führen, dass \ac{T2P} nochmals revolutioniert werden kann.