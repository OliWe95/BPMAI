\section{Zusammenfassung}
Diese Arbeit befasste sich mit der Anwendung von \ac{AI} Vorgehensweisen auf die Prozessmodell Generierung aus Beschreibungen in natürlicher, englischer Sprache. 
Die Betrachtung des State-Of-The Art Ansatzes hat gezeigt, dass sich mit der Weiterentwicklung der \ac{NLP} betrachteten Tools an manchen Stellen des Verfahrens neue Möglichkeiten ergeben.

\section{Ausblick}
Bei Friedrichs State-Of-The-Art Ansatz handelt es sich um einen regelbasierten Ansatz, der sich auch der Anwendung von Heuristiken bedient. Natürliche Sprache kann nicht vollständig an Regeln festgemacht werden, denn es bleibt immer ein Interpretationsspielraum. Folglich Bedarf es auch für die \ac{T2P} an menschlicher Interpretation. \par
Die Anwendung der Machine Learning Methode supervised learning wird bei ähnlichen Problemen bereits angewendet und verspricht effektive Resultate (\cite[vgl.][2]{BPMML}). Es Bedarf allerdings einer Datenbasis von inputs und outputs.