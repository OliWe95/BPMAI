\section{NLTK}

Das \ac{NLTK} ist eine Sammlung von Programmmodulen zur Verarbeitung von natürlicher Sprache. Ebenso wie beim Stanford CoreNLP handelt es sich um einen Standard Bibliothek, die grundlegende \ac{NLP} Aufgaben abdeckt. Das \ac{NLTK} steht unter der GLP open source licence und wurde für die Programmiersprache Pyhon entwickelt. Es greift auf diverse lexikalische Ressourcen wie WordNet, FrameNet aber auch Wikipedia zurück. Das \ac{NLTK} verfügt im Gegensatz zum Stanford CoreNLP über einfach handhabbare Visualisierungsmodule.


- Terminologie:
 Corpus: textkörper mit mehreren Sätzen und semantischen zusammenhängen
 Token: libguistische einheit (Wort, Satzzeichen oder Nummer)
 Satz: Geordnete Sequenz von token
 
 - nltk kann nicht in java, coreNLP aber in python genutzt werden
 - OpenNLTK?
