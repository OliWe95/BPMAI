\section{Business Process Modeling}
Viele Unternehmen werden heutzutage auf Grundlage von Geschäftsprozessen organisiert, was eine erheblichen EInfluss auf den Unternehmenserfolg hat . Für  Geschäftsprozesse existieren viele verschiedene Definitionen, wovon Davenports folgendermaßen lautet: 
\begin{quote}
"'structured, measured sets of activities designed to produce a specified output for a particular customer or market"'
\end{quote} \ac{BPM} bezeichnet weiterhin das allgemeine Verständnis derartiger Geschäftsprozesse. Die Relevanz dieser Modelle besteht darin, Optimierungs- und Automatisierungspotenzial der Geschäftsprzesse zu identifizieren. Weiterhin sind diese auch für die Softwaregestütze automatisierung der Prozesse hilfreich, da auch Software Entwicklung zunehmend durch Modelle getrieben wird. (\cite[vgl.][74]{BPM})\par
Es existieren verschiedene Standards, um einen Geschäftsprozess zu modellieren. Bespielsweise gibt es die \ac{BPMN}, EPKs oder Petri Netze. Die Auswahl der Modellierung hängt vom Use Case ab.