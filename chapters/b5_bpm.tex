\section{Business Process Modeling}
Viele Unternehmen werden heutzutage auf Grundlage von Geschäftsprozessen organisiert, was einen erheblichen Einfluss auf den Unternehmenserfolg hat.(\cite[vgl.][1]{BPM2}) Für den Begriff des Geschäftsprozesses existieren viele verschiedene Definitionen, wovon Davenports folgendermaßen lautet: 
\begin{quote}
"'structured, measured sets of activities designed to produce a specified output for a particular customer or market"' (\cite[5]{DAVENPORT})
\end{quote} \ac{BPM} bezeichnet weiterhin das allgemeine Verständnis derartiger Geschäftsprozesse. Der Verwendungszweck dieser Modelle besteht darin, Optimierungs- und Automatisierungspotenzial der Geschäftsprozesse zu identifizieren. Weiterhin sind diese auch für die Softwaregestütze Automatisierung der Prozesse hilfreich, da auch Software Entwicklung zunehmend durch Modelle getrieben wird. (\cite[vgl.][74]{BPM})\par
Es existieren viele verschiedene Standards an Techniken, um  Geschäftsprozesse zu modellieren. Bespiele sind die \ac{BPMN}, die \ac{EPK} und Petri Netze. Die einzelnen Modellierungstechniken unterscheiden sich in verschiedenen Eigenschaften, beispielsweise in ihrer allgemeinen Verständlichkeit oder ihrem Abstraktionsgrad, weshalb die Auswahl der Modellierungstechnik letztlich vom Use Case abhängt.(\cite[vgl.][75]{BPM2})