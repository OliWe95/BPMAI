\section{Business Process Modeling}
Die meisten Unternehmen organisieren sich heute entlang ihrer Geschäftsprozesse. Sie haben erkannt, dass eine geschäftsprozessorientierte Organisationsaufbau einen erheblichen positiven Einfluss auf den Unternehmenserfolg hat(\cite[vgl.][1]{BPM2}). Für den Begriff des Geschäftsprozesses existieren viele verschiedene Definitionen. Davenport definiert den Geschäftsprozess folgendermaßen: 
\begin{quote}
"'[A] structured, measured sets of activities designed to produce a specified output for a particular customer or market."' (\cite[5]{DAVENPORT})
\end{quote} \ac{BPM} bezeichnet weiterhin die Methoden der Darstellung derartiger Geschäftsprozesse. Die Kernaufgabe dieser Modelle besteht in der Identifikation von Optimierungs- und Automatisierungspotenzial der Geschäftsprozesse. Weiterhin sind diese auch für die softwaregestütze Automatisierung der Prozesse hilfreich, da auch Software Entwicklung zunehmend durch Modelle getrieben wird (\cite[vgl.][74]{BPM}).\par
Es existieren viele verschiedene Standards an Techniken, um  Geschäftsprozesse zu modellieren. Bespiele hierfür sind die \ac{BPMN}, die \ac{EPK} und Petri-Netze. Die einzelnen Modellierungstechniken unterscheiden sich in verschiedenen Eigenschaften, beispielsweise in ihrer allgemeinen Verständlichkeit oder ihrem Abstraktionsgrad, weshalb die Auswahl der Modellierungstechnik letztlich vom Use Case abhängt (\cite[vgl.][75]{BPM2}).
\par
Der Inhalt dieser Arbeit beschränkt sich nicht auf eine konkrete Modellierungsform, sondern konzsntriert sich auf die Erstellung einer Meta-Repräsentation einer Prozessbeschreibung, die dann in einem zweiten Schritt in ein beliebiges konkretes Modell übersetzt werden kann.