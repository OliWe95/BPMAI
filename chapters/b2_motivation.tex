\section{Umfeld und Motivation}

Zur Zeit wird das Feld der \ac{AI} in der Informatik immer wichtiger. Immer größere Datenmengen und steigende Rechenleistung schaffen Potenzial für Automatisierung in vielen verschiedenen Disziplinen. So gibt es im Bereich der Geschäftsprozesse Bedarf für AI. Diese Arbeit befasst sich detailliert mit einem solchen Anwendungsfall, der im Folgenden eingeführt wird.\par
Bei der Optimierung von Prozessen ist die Ist-Analyse des vorherrschenden Zustands der zeitintensivste Vorgang. Nach Herbst müssen etwa 60\% der Zeit eines solchen Projekts allein hierfür aufgewendet werden (\cite[vgl.][1]{HERBST}). 
Weiterhin sind verschiedene Fähigkeiten für diese Analyse notwendig, zum einen das Wissen über den fachlichen Kontext und zum anderen das Wissen über die Methodik des Vorgehens. Daher können an dieser Stelle leicht Missverständnisse zwischen den Akteuren entstehen. Zusammenfassend ist die Zustandsanalyse von Prozessen mit hohem Aufwand an Zeit und Ressourcen verbunden.\par
Mit Methoden der \ac{AI} könnten jedoch die in vielen Unternehmen vorhandenen, unstrukturierten Beschreibungen von Prozessen zu einer automatischen Ist-Analyse dieser und somit zur Steigerung der Effizienz eingesetzt werden. Die weitere Vertiefung dieses Ansatzes und insbesondere die dafür notwendigen Funktionalitäten und Tools auf Seite der \ac{AI} stellen das Umfeld dieser Arbeit dar.



%Anwendungsbeispiele?
%\begin{itemize} 
%\item automatische Optimierung
%\end{itemize}
