\section{Umfeld und Motivation}

Zur Zeit gewinnen das Forschungsgebiet \textit{\ac{AI}} und dessen Teildisziplin \textit{Machine Learning} in der Informatik zunehmend an Relevanz. Immer größere Datenmengen und steigende Rechenleistung schaffen Potenzial für Automatisierung in vielen verschiedenen Disziplinen (\cite[vgl.][1]{AIRELEVANCE}). So gibt es auch im Bereich des Geschäftsprozessmanagements vielseitigen Bedarf für AI-Technologien (\cite[vgl.][1]{BPMAIRELEVANCE}).\par
Bei der Optimierung von Prozessen ist die Ist-Analyse des vorherrschenden Zustands der zeitintensivste Vorgang. Nach Herbst müssen etwa 60\% der Zeit eines solchen Projekts allein hierfür aufgewendet werden (\cite[vgl.][1]{HERBST}). 
Weiterhin sind verschiedene Fähigkeiten für diese Analyse notwendig: Zum einen das Wissen über den fachlichen Kontext und zum anderen das Wissen über die Methodik des Vorgehens. Daher können an dieser Stelle leicht Missverständnisse zwischen It- und Fachbereichsexperten entstehen (\cite[vgl.][1]{HERBST}). Zusammenfassend ist die Zustandsanalyse von Prozessen mit hohem Aufwand an Zeit und Ressourcen verbunden.\par
Durch Einsatz von \ac{AI} könnten jedoch die in vielen Unternehmen vorhandenen, unstrukturierten textuellen Beschreibungen von Prozessen automatisiert analysiert und in eine graphische Darstellung überführt werden. Durch diese effiziente Vorgehensweise könnten Zeit gespart und Fehler vermieden werden. Hierzu wird auf \textit{Natural Language Processing} Techniken und Tools zurückgegriffen. Die weitere Vertiefung dieses Ansatzes und insbesondere die dafür notwendigen Funktionalitäten und Tools auf Seite der \ac{AI} stellen das Umfeld dieser Arbeit dar.
