\section{Motivation}

Umfeld und Motivation?
Warum BPM und AI? Anwendungsbeispiele?
\begin{itemize} 
\item automatische Optimierung
\end{itemize}

Bei der Optimierung von Prozessen ist die Ist-Analyse des vorherrschenden Zustands der zeitintensivste Vorgang. Nach Herbst müssen etwa 60\% der Zeit eines solchen Projekts allein hierfür aufgewendet werden (\cite[vgl.][1]{HERBST}). Weiterhin sind die für diese Analyse notwendigen Fähigkeiten, also das Wissen über den fachlichen Kontext und die Methodik des Vorgehens, auf verschiedene Personen verteilt. Daher können leicht Missverständnisse entstehen. Zusammenfassend ist die Zustandsanalyse von Prozessen mit hohem Aufwand an Zeit und Ressourcen verbunden.\par
Mit Methoden der \ac{AI} könnten jedoch die in vielen Unternehmen vorhandenen, unstrukturierten Beschreibungen von Prozessen zu einer automatischen Ist-Analyse dieser und somit zur Steigerung der Effizienz eingesetzt werden. Die weitere Vertiefung dieses Ansatzes und insbesondere die dafür Notwendigen Funktionalitäten und Tools auf Seite der \ac{AI} stellen das Umfeld dieser Arbeit dar.

Motivation nach Friedrich:
\begin{itemize} 
\item 85\% der Daten in Unternehmen sind unstrukturiert (sehr alte Quelle von 2003!)
\item Die Menge an unstrukturierten Daten wächst schneller als Menge an strukturierten Daten

\end{itemize}
