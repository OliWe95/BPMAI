
\documentclass[a4paper,
12pt,footsepline,oneside,			
plainfootsepline,listof=totoc,headinclude,
footinclude]{scrreprt}
\usepackage{graphicx}
\usepackage[utf8]{inputenc}
\usepackage[ngerman]{babel}
\usepackage{geometry}
\usepackage{setspace}
\usepackage[nohyperlinks,printonlyused, withpage,  withpage]{acronym}
\usepackage{float}
\usepackage{tabularx}
\usepackage{wrapfig}
\usepackage{color}
\usepackage{selinput}
\usepackage{fancyvrb}
\usepackage{booktabs}
\usepackage{listings}
\usepackage{blindtext}
\usepackage{scrpage2}
\usepackage{caption}
\usepackage[section]{placeins}
 

\pagestyle{scrheadings}
\usepackage[backend=biber,style=apa]{biblatex}
\DeclareLanguageMapping{german}{german-apa}
\addbibresource{lit.bib}
\captionsetup{justification=raggedright,singlelinecheck=false}

%Seitenzahlen
\clearscrplain
\clearscrheadings
\ofoot[\pagemark]{\pagemark}

%Coding Environment
\definecolor{codegreen}{rgb}{0,0.6,0}
\definecolor{codegray}{rgb}{0.5,0.5,0.5}
\definecolor{codepurple}{rgb}{0.58,0,0.82}
\definecolor{backcolour}{rgb}{0.95,0.95,0.92}
\lstdefinestyle{mystyle}{
	basicstyle= \small\footnotesize\ttfamily,
    backgroundcolor=\color{backcolour},   
    commentstyle=\color{codegreen},
    keywordstyle=\color{magenta},
    numberstyle=\tiny\color{codegray},
    stringstyle=\color{codepurple},
    basicstyle=\footnotesize,
    breakatwhitespace=false,         
    breaklines=true,                 
    captionpos=b,                    
    keepspaces=true,                 
    numbers=left,                    
    numbersep=5pt,                  
    showspaces=false,                
    showstringspaces=false,
    showtabs=false,                  
    tabsize=2
}
\lstset{style=mystyle}

%Umbenennungen
\renewcommand*{\figurename}{Abbildung}
\renewcommand{\contentsname}{Inhaltsverzeichnis}
\renewcommand*{\bibname}{Literaturverzeichnis}
\renewcommand*{\listfigurename}{Abbildungsverzeichnis}

%Absatzformatierung
\renewcommand*\chapterheadstartvskip{\vspace*{-\topskip}}
\renewcommand*\chapterheadendvskip{%
  \vspace*{1\baselineskip plus .1\baselineskip minus .167\baselineskip}}
\parskip 5pt plus 1pt minus 1pt
%\addtolength{\parskip}{\baselineskip}
\parindent 0pt    
  

%Pageoffset
\geometry{
 left=3.5cm,
  right=2.5cm,
  top=2.2cm,
  bottom=3cm,
  bindingoffset=2mm
}

%line spacing
\setstretch{1.5}	


%custom commands
\newcommand{\ThesisTitle}{Text2Process - der Stanford Parser}

\newcommand{\Heading}[1]{ \begin{center} 
\textbf{#1} 
\newline
\end{center}}
 

%document structure
\begin{document}

\pagenumbering{Roman}
\begin{titlepage}
	\centering
%	\includegraphics[width=0.25\textwidth]{pictures/SAP_Logo.png}
	
	\includegraphics{pictures/dhbw_logo.png}
	\vspace{1cm}
	\par
	{\scshape\LARGE Fakultät Wirtschaft\par}
	\vspace{1cm}
	{\scshape\Large Studiengang Wirtschaftsinformatik\par}
	\vspace{1.5cm}
	{\large\bfseries \ThesisTitle \par}
	\vspace{2cm}
	{\Large Integrationsseminar\par}
	\vfill
	{ Im Rahmen der Prüfung zum Bachelor of Science (B. Sc) \par}
	\vfill
	
	\vfill
	
	\begin{center}
	\begin{tabularx}{\columnwidth}{XXl}
	Verfasser: &  \textsc{Jana Kuntz,} \\
	&\textsc{Oliver Weisenburger} \\
	Kurs: & \textsc{WWI15B1} \\
	Partnerunternehmen:: & \textsc{Daimler AG, SAP SE} \\
	Wissenschaftlicher Betreuer: &  \textsc{Prof. Dr. Thomas Freytag} 	\\
	Abgabedatum: & \textsc{08.01.2018} \\
\end{tabularx} 
 \end{center}

\end{titlepage}
\newpage\cleardoublepage
\Heading{Eidesstattliche Erklärung}
Wir versicheren hiermit, dass wir unsere Seminararbeit mit dem Thema:    \begin{quote}
"\ThesisTitle "
\end{quote} selbstständig verfasst und keine anderen als die angegebenen Quellen und Hilfsmittel benutzt haben. Wir versicheren zudem, dass die eingereichte elektronische Fassung mit der
gedruckten Fassung übereinstimmt.
Diese Arbeit wurde bisher in gleicher oder ähnlicher Form oder auszugsweise
noch keiner Prüfungsbehörde vorgelegt und auch nicht veröffentlicht.

\vspace{50pt} 
\noindent\rule{5cm}{.4pt}\hfill\rule{5cm}{.4pt}\par 
\noindent Datum, Ort  \hspace{7,4cm} Datum, Ort 
\par
\vspace{2cm}
\par
\noindent\rule{5cm}{.4pt}\hfill\rule{5cm}{.4pt}\par 
\noindent Jana Kuntz \hspace{7,4cm} Oliver Weisenburger
\newpage\cleardoublepage

\tableofcontents \newpage\cleardoublepage
\listoffigures  \newpage\cleardoublepage
\addcontentsline{toc}{chapter}{Abkürzungsverzeichnis}
\listoftables  \newpage\cleardoublepage

\chapter*{Abkürzungsverzeichnis}
\begin{acronym} 
  \acro{BPMN}{Business Process Modeling Notation}
\end{acronym}


\chapter{Einleitung}\pagenumbering{arabic}	


\input{chapters/b1_example}
\chapter{Hauptteil}


\chapter{Schluss}

\addcontentsline{toc}{chapter}{Literaturverzeichnis}
\printbibliography


\appendix     
\part*{Anhang} % Start the appendix part
   
    
\chapter{Anhang} 

\begin{table}[h!]
 
  \begin{tabular}{|l|l|l|} 
      \toprule
    & Tag & Bedeutung\\
    \midrule

     1. & CC & Coordinating conjunction \\
	 2. & CD & Cardinal number\\
	 3. & DT & Determiner \\
	4.&  EX &	Existential there\\
	5.&	FW &	Foreign word\\
	6.&	IN &	Preposition or subordinating conjunction\\
	7.&	JJ &	Adjective\\
	8.&	JJR &	Adjective, comparative\\
	9.&	JJS &	Adjective, superlative\\
	10.& LS &	List item marker\\
	11.& MD &	Modal\\
	12.& NN &	Noun, singular or mass\\
	13.& NNS &	Noun, plural\\
	14.& NNP &	Proper noun, singular\\
	15.& NNPS &	Proper noun, plural\\
	16.& PDT & Predeterminer\\
	17.& POS & Possessive ending\\
	18. & PRP & Personal pronoun \\
	19.& PRP\$ & Possessive pronoun\\
	20.& RB & Adverb\\
	21.& RBR & Adverb, comparative\\
	22.& RBS & Adverb, superlative\\
	23.& RP & Particle\\
	24.& SYM & Symbol\\
	25.& TO & to\\
	26.& UH & Interjection\\
	27.& VB & Verb, base form\\
	28.& VBD & Verb, past tense\\
	29.& VBG & Verb, gerund or present participle\\
	30.& VBN & Verb, past participle\\
	31.& VBP & Verb, non-3rd person singular present\\
	32.& VBZ & Verb, 3rd person singular present\\
	33.& WDT & Wh-determiner\\
	34.& WP & Wh-pronoun\\
	35.& WP\$ & Possessive wh-pronoun\\
	36.& WRB & Wh-adverb	\\
	
	\bottomrule
  \end{tabular}
  \caption{Liste aller POS-Tags der Penn Treebank}
  \label{table:POSTAGS}
\end{table}

\begin{table}[h!]
 
  \begin{tabular}{|l|l|l|l|}
      \toprule
    & Tag & Bedeutung\\
    \midrule
   AUX & link between a content verb and an auxiliary verb, e.g. 'Reagan has died' & aux(died,has)\\
       AUXPASS & link between a passive participle and a passive auxiliary, e.g. 'Kennedy has been killed' & auxpass(killed,been)\\
       COP & link between a predicative content word and its copula, e.g. 'Bill is big' & cop(big,Bill), 'Bill is an honest man' & cop(man,is)\\
   CONJ & link between two (content) words connected by a conjunction, e.g. 'Bill is big and honest' & conj(big,honest)\\
   CC & link between a content word and a conjunction, e.g. 'Bill is big and honest' & cc(big,and)\\
   ARG & link between a verb and one of its arguments, e.g. 'Clinton defeated Dole' & arg(defeated,Clinton), arg(defeated,Dole)\\
       SUBJ & link between a verb and its subject, e.g. 'Clinton defeated Dole' & subj(defeated,Clinton), 'what she said is untrue' & subj(is,what she said)\\
           NSUBJ & link between a verb and an NP subject, e.g. 'Clinton defeated Dole' & nsubj(defeated,Clinton)\\
               NSUBJPASS & link between a passive participle and an NP surface subject, e.g. 'Dole was defeated by Clinton' & nsubjpass(defeated,Dole)\\
           CSUBJ & link between a verb and a CP subject, e.g. 'what she said makes sense' & csubj(makes,said), 'what she said is untrue' & ccsubj(untrue,said)\\
               CSUBJPASS & link between a passive participle and a CP surface subject, e.g. 'that she lied was suspected by everyone' & csubjpass(suspected,lied)\\
       COMP & link between a verb and its complement, e.g. 'she gave me a raise' & comp(gave,me), comp(gave,raise); 'I like to swim' & comp(like,swim)\\
           OBJ & link between a verb and one of its objects, e.g. 'she gave me a raise' & obj(gave,me), obj(gave,raise)\\
               DOBJ & link between a verb and one of its accusative objects, e.g. 'she gave me a raise' & dobj(gave,raise)\\
               IOBJ & link between a verb and its dative object, e.g. 'she gave me a raise' & iobj(gave,me)\\
               POBJ & link between a preposition and its object, e.g. 'on the chair' & pobj(on,chair)\\
           PCOMP & link between a preposition and a verb which heads its complement CP or VP, e.g. 'information on whether users are at risk' & pcomp(on,are), 'they heard about you missing classes' & pcomp(about,missing)\\
           ATTR & link between a verb like 'be/seem/appear' and its complement\\
           CCOMP & link between a verb or adjective and a CP complement (finite or remnant subjunctive), e.g. 'he says that you like to swim' & ccomp(says,like), 'I am certain that he did it' & ccomp(certain,did)\\
           XCOMP & link between a verb or adjective and a (controlled) VP complement, e.g. 'I like to swim' & xcomp(like,swim), 'I am ready to leave' & xcomp(ready,leave)\\
           COMPL(m) & link between a subordinate verb in a complement clause and the 'that' complementiser that introduces it, e.g. 'he says that you like to swim' & complm(like,that)\\
           MARK & link between a subordinate verb in an adverbial clause and the subordinating conjunction (i.e. marker) that introduces it, e.g. 'after insurgants launched simultaneous attacks' & mark(launched,after)\\
           REL & link between a verb in a relative clause and the head of the relative pronoun phrase which introduces it, e.g. 'the man that you love' & rel(love,that), 'the man whose wife you love' & rel(love,wife)\\
           ACOMP & link between a verb and an adjective complement, e.g. 'she looks very beautiful' & acomp(looks,beautiful)\\
       AGENT & link between a passive participle and the by&PP introducing its agent, e.g. 'The man has been klled by the police' & agent(killed,police)\\
   REF & link between a noun and a relative pronoun introducing a relative clause, e.g. 'the book which you bought' & ref(book,which)\\
   EXPL & link between a verb and an expletive 'there' subject, e.g. 'there is a statue in the corner' & expl(is,there)\\
   MOD & link between a verb and one of its modifiers, e.g. 'last night I swam in the pool' & mod(swam,in the pool), mod(swam,last night)\\
   SDEP & semantic dependent\\
       XSUBJ & link between a controlled verb and its controlling subject, e.g. 'Tom likes to eat fish' & xsubj(eat,Tom)\\
   PRED & link between a subject and its predicate, e.g. 'Reagan died' & pred(Reagan,died)\\
   PUNC & link between a word and a punctuation marker, e.g. 'Go home!' & punc(Go,!)\\

	\bottomrule
  \end{tabular}
  \caption{Liste aller DEP-Tags des Stanford Parsers}
  \label{table:DEPTAGS}
\end{table}



\end{document}
