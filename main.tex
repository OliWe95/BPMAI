
\documentclass[a4paper,
12pt,footsepline,oneside,			
plainfootsepline,listof=totoc,headinclude,
footinclude]{scrreprt}
\usepackage{graphicx}
\usepackage[utf8]{inputenc}
\usepackage[ngerman]{babel}
\usepackage{geometry}
\usepackage{setspace}
\usepackage[nohyperlinks,printonlyused, withpage,  withpage]{acronym}
\usepackage{float}
\usepackage{tabularx}
\usepackage{wrapfig}
\usepackage{color}
\usepackage{selinput}
\usepackage{fancyvrb}
\usepackage{booktabs}
\usepackage{listings}
\usepackage{blindtext}
\usepackage{scrpage2}
\usepackage{caption}
\usepackage[section]{placeins}
 

\pagestyle{scrheadings}
\usepackage[backend=biber,style=apa]{biblatex}
\DeclareLanguageMapping{german}{german-apa}
\addbibresource{lit.bib}
\captionsetup{justification=raggedright,singlelinecheck=false}

%Seitenzahlen
\clearscrplain
\clearscrheadings
\ofoot[\pagemark]{\pagemark}

%Coding Environment
\definecolor{codegreen}{rgb}{0,0.6,0}
\definecolor{codegray}{rgb}{0.5,0.5,0.5}
\definecolor{codepurple}{rgb}{0.58,0,0.82}
\definecolor{backcolour}{rgb}{0.95,0.95,0.92}
\lstdefinestyle{mystyle}{
	basicstyle= \small\footnotesize\ttfamily,
    backgroundcolor=\color{backcolour},   
    commentstyle=\color{codegreen},
    keywordstyle=\color{magenta},
    numberstyle=\tiny\color{codegray},
    stringstyle=\color{codepurple},
    basicstyle=\footnotesize,
    breakatwhitespace=false,         
    breaklines=true,                 
    captionpos=b,                    
    keepspaces=true,                 
    numbers=left,                    
    numbersep=5pt,                  
    showspaces=false,                
    showstringspaces=false,
    showtabs=false,                  
    tabsize=2
}
\lstset{style=mystyle}

%Umbenennungen
\renewcommand*{\figurename}{Abbildung}
\renewcommand{\contentsname}{Inhaltsverzeichnis}
\renewcommand*{\bibname}{Literaturverzeichnis}
\renewcommand*{\listfigurename}{Abbildungsverzeichnis}

%Absatzformatierung
\renewcommand*\chapterheadstartvskip{\vspace*{-\topskip}}
\renewcommand*\chapterheadendvskip{%
  \vspace*{1\baselineskip plus .1\baselineskip minus .167\baselineskip}}
\parskip 5pt plus 1pt minus 1pt
%\addtolength{\parskip}{\baselineskip}
\parindent 0pt    
  

%Pageoffset
\geometry{
 left=3.5cm,
  right=2.5cm,
  top=2.2cm,
  bottom=3cm,
  bindingoffset=2mm
}

%line spacing
\setstretch{1.5}	


%custom commands
\newcommand{\ThesisTitle}{Text2Process - der Stanford Parser}

\newcommand{\Heading}[1]{ \begin{center} 
\textbf{#1} 
\newline
\end{center}}
 

%document structure
\begin{document}

\pagenumbering{Roman}
\begin{titlepage}
	\centering
%	\includegraphics[width=0.25\textwidth]{pictures/SAP_Logo.png}
	
	\includegraphics{pictures/dhbw_logo.png}
	\vspace{1cm}
	\par
	{\scshape\LARGE Fakultät Wirtschaft\par}
	\vspace{1cm}
	{\scshape\Large Studiengang Wirtschaftsinformatik\par}
	\vspace{1.5cm}
	{\large\bfseries \ThesisTitle \par}
	\vspace{2cm}
	{\Large Integrationsseminar\par}
	\vfill
	{ Im Rahmen der Prüfung zum Bachelor of Science (B. Sc) \par}
	\vfill
	
	\vfill
	
	\begin{center}
	\begin{tabularx}{\columnwidth}{XXl}
	Verfasser: &  \textsc{Jana Kuntz,} \\
	&\textsc{Oliver Weisenburger} \\
	Kurs: & \textsc{WWI15B1} \\
	Partnerunternehmen:: & \textsc{Daimler AG, SAP SE} \\
	Wissenschaftlicher Betreuer: &  \textsc{Prof. Dr. Thomas Freytag} 	\\
	Abgabedatum: & \textsc{08.01.2018} \\
\end{tabularx} 
 \end{center}

\end{titlepage}
\newpage\cleardoublepage
\Heading{Eidesstattliche Erklärung}
Wir versicheren hiermit, dass wir unsere Seminararbeit mit dem Thema:    \begin{quote}
"\ThesisTitle "
\end{quote} selbstständig verfasst und keine anderen als die angegebenen Quellen und Hilfsmittel benutzt haben. Wir versicheren zudem, dass die eingereichte elektronische Fassung mit der
gedruckten Fassung übereinstimmt.
Diese Arbeit wurde bisher in gleicher oder ähnlicher Form oder auszugsweise
noch keiner Prüfungsbehörde vorgelegt und auch nicht veröffentlicht.

\vspace{50pt} 
\noindent\rule{5cm}{.4pt}\hfill\rule{5cm}{.4pt}\par 
\noindent Datum, Ort  \hspace{7,4cm} Datum, Ort 
\par
\vspace{2cm}
\par
\noindent\rule{5cm}{.4pt}\hfill\rule{5cm}{.4pt}\par 
\noindent Jana Kuntz \hspace{7,4cm} Oliver Weisenburger
\newpage\cleardoublepage

\tableofcontents \newpage\cleardoublepage
\listoffigures  \newpage\cleardoublepage
\addcontentsline{toc}{chapter}{Abkürzungsverzeichnis}
\listoftables  \newpage\cleardoublepage

\chapter*{Abkürzungsverzeichnis}
\begin{acronym} 
  \acro{BPMN}{Business Process Modeling Notation}
  \acro{OWL}{Web Ontology Language}
\end{acronym}


\chapter{Einleitung}\pagenumbering{arabic}	
\input{chapters/b1_example} %Nur Latex Beispiele
\input{chapters/b2_nlp}
\input{chapters/b3_motivation}
\input{chapters/b4_petribpm}
\input{chapters/b5_ziel}

\chapter{State-of-the-Art}
\input{chapters/b6_friedrich}
\section{Syntaktische und Semantische Analyse}

\chapter{NLP Tools}
\section{Wordnet}


\subsection{Ontologien im NLP}

Bei der zwischenmenschlichen Kommunikation werden Informationen in codierter Form über das Medium der Sprache übertragen. Für den Erfolg des Informationstausches ist die richtige Decodierung der abstrakten Spracheinformationen, das heißt die richtige Verknüpfung eines Wortes mit dessen Bedeutung, entscheidend. 
\begin{quote} \textit{"`In order to effectively exchange information, agents need to share a lexicon of words as well as to access the world model(s) underlying the lexicon."'} (\cite[vgl.][1]{OLTRAMANI})\end{quote} 
Die Kommunikationspartner müssen sowohl über denselben Wortschatz als auch über ein gemeinsames Weltverständnis (Welt Modell) verfügen. Dies gilt auch für die Mensch-Maschine Kommunikation. Ein Weltmodell kann beispielsweise mittels einer Ontologie in maschinenlesbarer Form abgebildet werden. Eine Ontologie ist in der Informatik definiert als System von Informationen mit logischen Relationen (\cite[vgl.][1]{DUDEN}). Eingeführt wurde der Begriff in de 1970er Jahren auf dem Forschungsgebiet der künstlichen Intelligenz zur Modellierung von Wissen. Ziel von digitalen Ontolgien ist es, Wissensdomänen in maschinenlesbarer Form verfügbar zu machen. (\cite[vgl.][7]{TACKE})Somit sind Ontologien eine zentrale Komponente in Wissenssystemen. 
\par
Das \textit{Princeton WordNet} ist eine solche Ontologie, also eine als semantisches Netz organisierte, lexikalische Ressource. Die Entwicklung von WordNet begann bereits 1985 and der Universität Princeton. 1993 wurde die erste Version veröffentlicht. Die aktuelle Version, das \textit{WordNet 3.0}, enthält etwa 155.000 manuell verfasste Einträge in englischer Sprache. (\cite[vgl.][1]{PRINCETON}) Die aktuelle Version folgt den Linked Open Data Principles, und somit auch den Grundsätzen des Semantic Web. \textit{„Linked Open Data (LOD) is Linked Data which is released under an open licence, which does not impede its reuse for free.“}(\cite[vgl.][1]{BERNERS_LEE}). Die freie Verfügbarkeit und die Linked-Data-Struktur haben maßgeblich zur weiten Verbreitung von WordNet beigetragen. Als Metadatenformat in Wordnet wurde die weit verbreitete, standardisierte \textit{Web Ontology Language} \ac{OWL} eingesetzt. Im \ac{NLP} wird auf WordNet für die Erstellung der semantischen Annotationen benötigt. Es dient als essentielle Ressource für überwacht lernende Algorithmen.
\par

\subsection{Aufbau und Struktur}

In WordNet werden Wörter mit gleicher oder nahezu identischer Bedeutung in Synsets gruppiert. Der Begriff ist ein Neologismus aus \textit{synonym} (dt. Synonym) und \textit{set} (dt. Menge, Zusammenstellung).Die Synsets werden über Kanten verknüpft, die bestimmte semantische Relationen repräsentieren. So entsteht ein gerichteter azyklischer Graph (\cite[vgl.][12]{OLTRAMANI}), der dem Aufbau des neuronalen Netzes im menschlichen Gehirn ähnelt.Grundsätzlich setzt sich WordNet aus vier Datenbanken für die jeweiligen Wortarten Nomen, Verben, Adjektive und Adverben zusammen. Jede dieser Datenbanken beinhaltet miteinander verknüpfte Synsets. Hierbei wird jeweils auch die Arte der Verknüpfung bzw. der semantischen Relation angegeben. Abhängig von der Wortart sind verschiedene Arten von Relationen möglich. \ref{table:table2} zeigt die Relationen von Nomen und Verben in WordNet geordnet nach Häufigkeit.
\par
\begin{table}[h!]
  \centering
  \begin{tabular}{ccccc} %Hier muss für jede Spalte ein c hin -> Dump wegen mismatch von definierter und auftretender Spaltenanzahl
    \toprule
     Rang & Relationen der Nomen & Anteil  & Relationen der Verben & Anteil \\
    \midrule
    1 & Hyponyme/Hypernyme: & 45.5\% & Abgeleitete Form:    & 55.4\% \\
    2 & Abgeleitete Form:   & 22.4\% & Troponyme/Hypernyme: & 31.7\% \\
    3 & Meronyme/Holonymy:  & 13.3\% & Verbgruppe:          & 4.2\%  \\
    4 & Wissensgebiet:      & 9.1\%  & Wissensgebiet:       & 3.0\%  \\
    5 & Typ/Instanz: 		& 5.1\%  & Antonymy:			& 2.6\%  \\
    6 & Pertainyme: 		& 2.9\%  & Siehe auch: 		    & 1.4\%  \\
    7 & Antonyme: 			& 1.3\% & Folgebeziehung: 	    & 1.0\%  \\
	8 & Attribut: 			& 0.4\%  & Ursache: 			& 0.5\%  \\
	9 & Partizip: 	        & 0.2\%  & -     	            & -  	 \\
	\bottomrule
  \end{tabular}
  \caption{Relationen in WordNet (\cite[vgl.][9]{MAZIARZ})}
  \label{table:table2}
\end{table}
\par

Der bei Nomen am häufigsten auftretende Beziehungstyp ist das Hyponym/Hypernym. Diese Beziehung strukturiert die Synsets hierarchisch in übergordnete Elter-Synsets (Hypernym) und untergeordnete Kind-Synsets. Es handelt sich also um eine IS-A-Beziehung. Das äquivalent zu Hyponymen in der Kategorie Verben wird als Troponym bezeichnet. Abgleitete Formen verbinden Synsets über Wortarten-Grenzen hinweg. Beispielsweise ist das Verb \textit{run} mit dem Nomen \textit{run} über die Beziehung der abgeleiteten Form verbunden. Ein weiterer häufiger Beziehungstyp sind Meronyme- und Holonyme-Beziehungen. Sie stellen die Part-Of-Beziehung (Meronym) und deren Gegenstück dar.

\par
\subsection{Word Sense Disambiguation}

Die meisten Worte einer Sprache haben mehrere Bedeutungen. Die richtige Wortbedeutung ergibt sich folglich erst aus dem Kontext der Wortverwendung. Um die Wortbedeutung maschinell zu erfassen, sind ein \textit{Sense Inventory} und ein klassifizierender Algorithmus notwendig. Bei einem solchen Sense Inventory handelt es sich um eine Informationsquelle, in der die verschiedenen Wortbedeutungen einsehbar sind. WordNet kann diese Rolle übernehmen. Ein mehrdeutiges Wort ist in WordNet Mitglied in verschiedenen Synsets, aus deren hypernymen Synsets sich dann die verschiedenen Wortbedeutungen ableiten lassen. \ref{fig:WSD} zeigt einen Beispielsatz, indem die Bedeutung des englischen Wortes \textit{coffee} ermittelt werden soll.
\par 

\begin{wrapfigure}{r}{15cm}
\includegraphics[width=15cm]{pictures/WSD.png}
\caption{WSD Klassifikation mit WordNet (eigene Darstellung)}
\label{fig:WSD}
\end{wrapfigure}

\par
Der Klassifikationsalogrithmus, hier als Blackbox dargestellt, nutzt die hierarchische Struktur von WordNet um die verschiedenen Bedeutungen nachzuschlagen. Daraufhin wählt er die wahrscheinlichste Bedeutung im Satzzusammenhang aus. In diesem Fall wählt er korrekt die Bedeutung \textit{beverage} aus.
\par
Somit ist Word Sense Disambiguation nichts anderes als ein Klassifikationsproblem, das basierend auf gewissen Eingabeparametern eine Einordnung vornimmt. Prinzipiell kann jeder generische, maschinell lernende Klassifikationsalgorithmus zur Lösung herangezogen werden \cite[vgl.][326]{YAROWSKY}. Als am verlässlichsten stellen jedoch Kombinationen verschiedener Ranking-Algorithem heraus. Die geringe Menge an verfügbaren Trainingsdaten stellt hierbei jedoch immernoch ein Problem dar.


\subsection{Java APIs}

Um WordNet direkt in eigene Softwareprojekte einbinden zu können, stehen allein für die Programmiersprache Java 12 verschiedenen Bibliotheken zu Verfügung. Bei der Wahl einer geeigneten Bibliothek hängt von verschiedenen Faktoren ab. Im Vergleich schneidet das \textit{Java WordNet Interface} \ac{JWI} für den allgemeinen Anwendungsfall am besten ab (\cite[vgl.][2]{FINLAYSON}). Auch gegenüber der weit verbreiteten \textit{Java WordNet Library} \ac{JWNL} bietet es nach Finlayson die folgenden Vorteile:

\begin{itemize} 
\item Vollumfänglicher Zugriff auf alle WordNet Ressourcen
\item Arbeitet direkt mit den WordNet-Files (keine Modifizierung notwendig)
\item Dateibasierte und In-Memory-basierte Implementierungsmöglichkeiten
\item Anzahl der instanziierten Dictionaries kann beliebig groß gewählt werden
\item Sehr hohe Performanz, speicherschonend und unabhängig von anderen Systemen
\item Keine config-Datei benötigt 
\item Ausführliche Dokumentation, aktiver Support und kontinuierliche Weiterentwicklung
\end{itemize}

Neben der direkten Einbindung von Wordnet über eine Java API, wird Wordnet auch von Standard-Software-Toolkits wie dem Standford CoreNLP oder dem NLTK als Informationsquelle genutzt.
\section{Stanford Parser}
\label{sec:stanfordparser}

Die Stanford \ac{NLP} Group, ein Team aus Software- und Linguistik Experten, entwickelt quelloffene Software zur Anwendung von \ac{NLP}. Resultat dieser Entwicklung ist eine Menge von Programmen, die jeweils eigenständige \ac{NLP} Probleme lösen und als Software mit eigener Distribution verfügbar sind. Darunter befindet sich der Stanford Parser. Dieser ist, neben anderen Modulen, aber auch Bestandteil der \textit{Stanford CoreNLP} Suite, einer seit 2010 offen verfügbaren, vereinigten Distribution der verschiedenen \ac{NLP} Komponenten mit einheitlicher Schnittstelle (\cite[vgl.][1 ff.]{STANFORDNLP}). Das grundlegende Prinzip des Stanford CoreNLPs besteht darin, den zu analysierenden Text in einzelne Elemente, also etwa Worte, zu zerlegen und diese mit Meta-Annotationen verschiedener funktionaler Komponenten zu versehen.\par
Dieses Kapitel soll die einzelnen Komponenten und ihre Funktionsweise grundlegend erläutern, insbesondere den Parser. Zu diesem Zweck wird zur Erläuterung der einiger Funktionen eine Analyse-Visualisierung folgenden Minimalbeispiels verwendet:
\begin{quote}
"'The student writes his paper at the DHBW. After he is finished, he hands it in."'
\end{quote} Diese Visualisierung wird jeweils anhand eines Web-Tools der Stanford \ac{NLP} Group vorgenommen.\footnote{Erreichbar unter: http://nlp.stanford.edu:8080/corenlp/} Weiterhin soll die Verwendung des Core NLP Toolkits mit Java thematisiert werden. 

\subsection{Part-of-Speech}
\label{subsec:pos}

\begin{wrapfigure}{r}{7cm}
\includegraphics[width=7cm]{pictures/POS.png}
\caption{Visualisierung der POS am Beispiel}
\label{fig:POS}
\end{wrapfigure}

Als \ac{POS} wird die Art eines Wortes bezeichnet. Ein Wort kann beispielsweise ein Nomen, Verb oder Adjektiv sein. Der Stanford Parser verfügt über eine Funktionalität, die \ac{POS} Annotationen zu den einzelnen Worten des analysierten Texts hinzufügt.\par
Zu diesem Zweck werden vorhergehende und folgende Wörter im jeweiligen Satz betrachtet. Realisiert wird diese Funktionalität über ein Dependency-Network (\cite[vgl.][1]{POSTAGGER}), das auf die \textit{Penn Treebank} angewandt wird. Dabei handelt es sich um eine Datenbank von syntaktisch analysiertem Text. Abbildung \ref{fig:POS} zeigt das mit POS-Tags versehene Beispiel. Jedes einzelne Wort, wird einer Wortart anhand einer Abkürzung zugeordnet. Beispielsweise steht der dem Wort "'student"' zugeordnete Tag "'NN"' für \textit{Noun}. Der POS-Tagger unterscheidet die 36 Wortarten verschiedenen Wortarten der Penn Treebank (\cite[vgl.][3]{PENNTREEBANK}), die in Tabelle \ref{table:POSTAGS} vollständig aufgelistet sind.

\subsection{Parser}
\label{subsec:parser}
\begin{wrapfigure}{r}{7cm}
\includegraphics[width=7cm]{pictures/Parser.png}
\caption{Visualisierung der Erweiterten Abhängikeiten am Beispiel}
\label{fig:ENHDEPS}
\end{wrapfigure}
Der Stanford Parser baut auf den POS-Tags auf und liefert eine nochmals erweiterte syntaktische Analyse des Textes. Es werden zusätzlich zusammengehörige Phrasen im Satz ermittelt. Dafür werden wiederum zwei unterschiedliche Repräsentationen unterstützt (\cite[vgl.][4]{STANFORDNLP}). Die Analyse kann zum einen durch Abhängigkeiten, zum anderen durch den Aufbau wiedergegeben werden. Abhängigkeiten eines bestimmten Typs werden zwischen zwei Worten gebildet. Die Notation der Analyse nach dem Aufbau erfolgt hingegen anhand einer Baumstruktur. Der Text wird in Phrasen-Typen eingeteilt und die einzelnen Worte diesen zugeordnet.\par Es existieren verschiedene Implementierungsstrategien für den Stanford Parser. Im CoreNLP Kit wird ein Ansatz anhand eines Neuronalen Netzes verfolgt, der entsprechend auf Wahrscheinlichkeiten basiert. Dieser zeichnet sich vor Allem durch die Realisierung performanter Ausführungszeiten bei einer präzisen Vorhersagegenauigkeit aus (\cite[vgl.][8]{DEPPARSER}).\par
Abbildung \ref{fig:ENHDEPS} zeigt die Syntaktische Analyse anhand der Abhängigkeiten innerhalb der beiden Beispielsätze. So existiert etwa eine Abhängigkeit des Typs det (determiner) vom Wort "'student"' zum Wort "'the"'. Die Schlussfolgerung ist, dass diese beiden Worte eine inhaltliche Phrase bilden.

\subsection{Named-Entity-Recognition}
\label{subsec:ner}
\begin{wrapfigure}{r}{7cm}
\includegraphics[width=7cm]{pictures/NER.png}
\caption{Visualisierung der NER am Beispiel}
\label{fig:NER}
\end{wrapfigure}
Ein Bestandteil natürlicher Sprache ist die Erwähnung von Eigennamen etwa bestimmter Orte, Personen oder auch Organisationen. Daher verfügt das CoreNLP Kit über die \ac{NER}. Mit Hilfe dieser Funktionalität können derartige Wörter erkannt und in Bezug auf die Entität, welche sie bezeichnen,  klassifiziert werden. Zu diesem Zweck existieren die vier Kategorien person(PER), location(LOC), organization(ORG) und miscellaneous(MISC) (\cite[vgl.][4]{STANFORDNER}).\par
Beispielsweise könnte ein Wort mit der Annotation "'ORG"' versehen werden, was darauf hinweist, dass es sich dabei um den Namen einer Organisation handelt. So zeigt Abbildung \ref{fig:NER}, dass die Bezeichnung "'DHBW"' im Beispielsatz von der \ac{NER} als Organisation erkannt wird.\par
Zur Implementierung dieses Features wurde eine lexikalische Datenbasis verwendet, weswegen nicht alle Namen als solche erkannt werden können. Werden bestimmte, der \ac{NER} unbekannte Namen im Text erwartet, können diese auch regelbasiert anhand eines frei definierbaren Regulären Ausdrucks identifiziert werden (REGEXNER).

\subsection{Coreference Resolution}
\label{subsec:coref}
\begin{wrapfigure}{r}{7cm}
\includegraphics[width=7cm]{pictures/coref.png}
\caption{Visualisierung der Coreference Resolution am Beispiel}
\label{fig:COREF}
\end{wrapfigure}
Ein Text in natürlicher Sprache bezieht sich auf Entitäten nicht nur anhand deren eigentlicher Bezeichnung. In der Regel werden diese Bezeichnungen in derartigen Bezügen auch durch Artikel oder Pronomina ersetzt. Dies kann insbesondere auch satzübergreifend geschehen. Das CoreNLP Kit ermöglicht über die Coreference Resolution eine Auflösung der Beziehungen zwischen den bezeichneten Entitäten und ihren jeweiligen substitutiven Elementen.\par
Abbildung \ref{fig:COREF} zeigt die Bezüge der Pronomina zu ihren jeweiligen Entitäten über beide Sätze des Beispiels hinweg. 

\subsection{Sentiment Analysis}
\label{subsec:sentiment}
Neben reinem Inhalt verfügt ein Satz in natürlicher Sprache auch über eine Stimmung. Diese äußert sich vorrangig anhand der Auswahl der Worte und deren Konnotation. Das CoreNLP Kit bietet daher die Sentiment Analysis. Hiermit können Sätze bezüglich ihrer Stimmung als "'negativ"', "'neutral"' oder "'positiv"' annotiert werden.\par
Umgesetzt wird diese Funktionalität unter Anwendung eines rekursiven Neuronalen Netzes, also eines Deep Learning Algorithmus. Zum Training des Netzes wurde die "'Stanford Sentiment Treebank"' verwendet,eine Sammlung von Begriffen versehen mit der jeweiligen Konnotation (\cite[vgl.][1]{SOCHERSENTIMENT}).

\subsection{Java API}
\label{subsec:corenlpjava}
Alle Funktionen des Stanford CoreNLP Toolkits können unter einem Einheitlichen \ac{API} adressiert werden, welches ursprünglich für die Verwendung unter Java konzipiert wurde. Jedoch existieren auch Implementierungen für andere Sprachen, wie etwa Python, Ruby oder Scala (\cite[vgl.][3]{STANFORDNLP}). Es soll weiterhin die grundlegende Funktionsweise dieser \ac{API} erläutert werden.\par
Die Basis dafür bildet ein Annotation-Objekt, welches reinen Text als Eingabe akzeptiert und nach Ausführung einen annotierten Text ausgibt. Ein mögliches Ausgabeformat für den Annotierten Text ist unter anderem XML. Die funktionalen Module des CoreNLP Kits, wovon die meisten in diesem Kapitel bereits erkäutert wurden, werden über Annotater repräsentiert. Diese werden sequentiell auf den Text angewendet und fügen Annotationen hinzu. Im Konstruktor des Annotation-Objekts können die gewünschten Annotater über einen String spezifiziert werden (\cite[vgl.][1]{STANFORDNLP}).\par
Weiterhin besteht die Möglichkeit eigene Logik anhand eigener Annotator auszuführen. Hierzu muss ein Interface implementiert werden und der Name der Klasse angegeben. Die Instanzierung erfolgt wie bei den Standard Klassen durch den String des Annotation-Konstruktors über class-reflection (\cite[vgl.][4]{STANFORDNLP}).

\subsection{Weitere Standard NLP-Tookits}

Neben dem Stanford CoreNLP gibt es noch zwei weitere weit verbreitete Standards. Das \ac{NLTK} ist eine Sammlung von Programmmodulen zur Verarbeitung von natürlicher Sprache. Ebenso wie beim Stanford CoreNLP handelt es sich um einen Standard Bibliothek, die grundlegende \ac{NLP} Aufgaben abdeckt. Das \ac{NLTK} steht unter der \textit{GLP Open Source Licence} und wurde für die Programmiersprache Pyhon entwickelt. Es greift auf diverse lexikalische Ressourcen wie WordNet, FrameNet aber auch Wikipedia zurück. Das \ac{NLTK} verfügt im Gegensatz zum Stanford CoreNLP über einfach handhabbare Visualisierungsmodule. Das StanfordCoreNLP kann zwar in Python, das NLTK aber nicht in Java verwendet werden.
\par
Eine weitere Alternative stelle das Apache Projekt OpenNLTK dar. Die Java-Bibliothek ist Machine Learning basiert und erfüllt ebenfalls die gängigen NLP Aufgaben.



\chapter{Text2Process}

\chapter{Zusammenfassung und Ausblick}


\chapter{Schluss}

\addcontentsline{toc}{chapter}{Literaturverzeichnis}
\printbibliography


\appendix     
\part*{Anhang} % Start the appendix part
   
    
\chapter{Anhang} 

\begin{table}[h!]
 
  \begin{tabular}{|l|l|l|} 
      \toprule
    & Tag & Bedeutung\\
    \midrule

     1. & CC & Coordinating conjunction \\
	 2. & CD & Cardinal number\\
	 3. & DT & Determiner \\
	4.&  EX &	Existential there\\
	5.&	FW &	Foreign word\\
	6.&	IN &	Preposition or subordinating conjunction\\
	7.&	JJ &	Adjective\\
	8.&	JJR &	Adjective, comparative\\
	9.&	JJS &	Adjective, superlative\\
	10.& LS &	List item marker\\
	11.& MD &	Modal\\
	12.& NN &	Noun, singular or mass\\
	13.& NNS &	Noun, plural\\
	14.& NNP &	Proper noun, singular\\
	15.& NNPS &	Proper noun, plural\\
	16.& PDT & Predeterminer\\
	17.& POS & Possessive ending\\
	18. & PRP & Personal pronoun \\
	19.& PRP\$ & Possessive pronoun\\
	20.& RB & Adverb\\
	21.& RBR & Adverb, comparative\\
	22.& RBS & Adverb, superlative\\
	23.& RP & Particle\\
	24.& SYM & Symbol\\
	25.& TO & to\\
	26.& UH & Interjection\\
	27.& VB & Verb, base form\\
	28.& VBD & Verb, past tense\\
	29.& VBG & Verb, gerund or present participle\\
	30.& VBN & Verb, past participle\\
	31.& VBP & Verb, non-3rd person singular present\\
	32.& VBZ & Verb, 3rd person singular present\\
	33.& WDT & Wh-determiner\\
	34.& WP & Wh-pronoun\\
	35.& WP\$ & Possessive wh-pronoun\\
	36.& WRB & Wh-adverb	\\
	
	\bottomrule
  \end{tabular}
  \caption{Liste aller POS-Tags der Penn Treebank}
  \label{table:POSTAGS}
\end{table}

\begin{table}[h!]
 
  \begin{tabular}{|l|l|l|l|}
      \toprule
    & Tag & Bedeutung\\
    \midrule
   AUX & link between a content verb and an auxiliary verb, e.g. 'Reagan has died' & aux(died,has)\\
       AUXPASS & link between a passive participle and a passive auxiliary, e.g. 'Kennedy has been killed' & auxpass(killed,been)\\
       COP & link between a predicative content word and its copula, e.g. 'Bill is big' & cop(big,Bill), 'Bill is an honest man' & cop(man,is)\\
   CONJ & link between two (content) words connected by a conjunction, e.g. 'Bill is big and honest' & conj(big,honest)\\
   CC & link between a content word and a conjunction, e.g. 'Bill is big and honest' & cc(big,and)\\
   ARG & link between a verb and one of its arguments, e.g. 'Clinton defeated Dole' & arg(defeated,Clinton), arg(defeated,Dole)\\
       SUBJ & link between a verb and its subject, e.g. 'Clinton defeated Dole' & subj(defeated,Clinton), 'what she said is untrue' & subj(is,what she said)\\
           NSUBJ & link between a verb and an NP subject, e.g. 'Clinton defeated Dole' & nsubj(defeated,Clinton)\\
               NSUBJPASS & link between a passive participle and an NP surface subject, e.g. 'Dole was defeated by Clinton' & nsubjpass(defeated,Dole)\\
           CSUBJ & link between a verb and a CP subject, e.g. 'what she said makes sense' & csubj(makes,said), 'what she said is untrue' & ccsubj(untrue,said)\\
               CSUBJPASS & link between a passive participle and a CP surface subject, e.g. 'that she lied was suspected by everyone' & csubjpass(suspected,lied)\\
       COMP & link between a verb and its complement, e.g. 'she gave me a raise' & comp(gave,me), comp(gave,raise); 'I like to swim' & comp(like,swim)\\
           OBJ & link between a verb and one of its objects, e.g. 'she gave me a raise' & obj(gave,me), obj(gave,raise)\\
               DOBJ & link between a verb and one of its accusative objects, e.g. 'she gave me a raise' & dobj(gave,raise)\\
               IOBJ & link between a verb and its dative object, e.g. 'she gave me a raise' & iobj(gave,me)\\
               POBJ & link between a preposition and its object, e.g. 'on the chair' & pobj(on,chair)\\
           PCOMP & link between a preposition and a verb which heads its complement CP or VP, e.g. 'information on whether users are at risk' & pcomp(on,are), 'they heard about you missing classes' & pcomp(about,missing)\\
           ATTR & link between a verb like 'be/seem/appear' and its complement\\
           CCOMP & link between a verb or adjective and a CP complement (finite or remnant subjunctive), e.g. 'he says that you like to swim' & ccomp(says,like), 'I am certain that he did it' & ccomp(certain,did)\\
           XCOMP & link between a verb or adjective and a (controlled) VP complement, e.g. 'I like to swim' & xcomp(like,swim), 'I am ready to leave' & xcomp(ready,leave)\\
           COMPL(m) & link between a subordinate verb in a complement clause and the 'that' complementiser that introduces it, e.g. 'he says that you like to swim' & complm(like,that)\\
           MARK & link between a subordinate verb in an adverbial clause and the subordinating conjunction (i.e. marker) that introduces it, e.g. 'after insurgants launched simultaneous attacks' & mark(launched,after)\\
           REL & link between a verb in a relative clause and the head of the relative pronoun phrase which introduces it, e.g. 'the man that you love' & rel(love,that), 'the man whose wife you love' & rel(love,wife)\\
           ACOMP & link between a verb and an adjective complement, e.g. 'she looks very beautiful' & acomp(looks,beautiful)\\
       AGENT & link between a passive participle and the by&PP introducing its agent, e.g. 'The man has been klled by the police' & agent(killed,police)\\
   REF & link between a noun and a relative pronoun introducing a relative clause, e.g. 'the book which you bought' & ref(book,which)\\
   EXPL & link between a verb and an expletive 'there' subject, e.g. 'there is a statue in the corner' & expl(is,there)\\
   MOD & link between a verb and one of its modifiers, e.g. 'last night I swam in the pool' & mod(swam,in the pool), mod(swam,last night)\\
   SDEP & semantic dependent\\
       XSUBJ & link between a controlled verb and its controlling subject, e.g. 'Tom likes to eat fish' & xsubj(eat,Tom)\\
   PRED & link between a subject and its predicate, e.g. 'Reagan died' & pred(Reagan,died)\\
   PUNC & link between a word and a punctuation marker, e.g. 'Go home!' & punc(Go,!)\\

	\bottomrule
  \end{tabular}
  \caption{Liste aller DEP-Tags des Stanford Parsers}
  \label{table:DEPTAGS}
\end{table}



\end{document}
